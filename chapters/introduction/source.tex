\documentclass[../../main.tex]{subfiles}
\begin{document}

%%%%%%%%%%%%%%%%%%%%%%%%%%%%%%%%%%%%%%%%%
%%%%%%%%%%%%%%%%%%%%%%%%%%%%%%%%%%%%%%%%%
%%%%%%%%%%%%%%%%%%%%%%%%%%%%%%%%%%%%%%%%%
\chapter{Introduction}


These notes serve as an introduction to the topic of elementary Business Statistics. 


%%%%%%%%%%%%%%%%%%%%%%%%%%%%%%%%%%%%%%%%%
%%%%%%%%%%%%%%%%%%%%%%%%%%%%%%%%%%%%%%%%%
\section{What is statistics?}

We can sum up elementary statistics like this. Elementary statistics tells us how to learn about a large \vocab{population} by studying \vocab{samples} of it. 

Think of a collection of objects that is so big that we just cannot examine each and every object in it. What we can do instead is take small samples from that collection, and then study those. Statistics tells us how to take what we learn from those samples, and use it to make estimates about the larger collection.

For example, suppose we want to know the average height of all public high school students in the United States. 

\begin{itemize}

  \item There are simply too many students here. Nobody could realistically go measure the height of every such student.
  
  \item But, we can select a sample of them (for example, we might randomly select 100 students from each state), and we can measure the heights of just those students. 
  
  \item Then, we can use statistics to take what we learn from that \emph{sample}, and then estimate what the average height of \emph{all} U.S. high school students is likely to be.

\end{itemize}

Of course, we have to be very careful when extrapolating from a sample to the larger group like this. The sample may not capture all the cases. For instance, we might (just by chance) select many of the tallest students from each state, and leave out many of the shortest students. Then the sample would be skewed. Statistics gives us mathematically precise ways to handle such problems, in a rigorous way.


%%%%%%%%%%%%%%%%%%%%%%%%%%%%%%%%%%%%%%%%%
%%%%%%%%%%%%%%%%%%%%%%%%%%%%%%%%%%%%%%%%%
\section{Sub-disciplines}

To study a sample and use that to make an estimate about a larger population, we need to take three steps:

\begin{itemize}

  \item We need to know how to actually understand and compute values from a data set that we collect from a sample. In other words, we need to know how to \emph{describe} a data set in statistically significant ways.
  
  \item If we measure something in a sample (for example, the height of the students), we then need to know what the probability is that what we see in the sample is reflected in the larger population. So we need to know how to calculate \emph{probabilities}.
  
  \item Finally, we need to use the sample data and our computed probabilities to make estimates about what must be true in the larger population, and then we need a way to test our estimates to see how accurate they are. In other words, we need to have a procedure that tells us how to \emph{infer} things about the larger population from the samples.

\end{itemize}

These three steps correspond to the three main sub-disciplines of business statistics that we will study:

\begin{itemize}

  \item \vocab{Descriptive statistics}. This is the set of concepts and techniques that are used to \emph{describe} statistical features of data sets.
    
  \item \vocab{Probability}. Probability is the mathematical study of probabilities or likelihood, i.e., calculating the odds.
  
  \item \vocab{Inferential statistics}. Inferential statistics is the set of concepts and techniques that are used to make \emph{estimates} about larger populations from samples.

\end{itemize}


%%%%%%%%%%%%%%%%%%%%%%%%%%%%%%%%%%%%%%%%%
%%%%%%%%%%%%%%%%%%%%%%%%%%%%%%%%%%%%%%%%%
\section{Good judgment}

Statistics is all about estimation. We study samples and then use that to make estimates about a larger population. So there is a certain sense in which statistics is not an exact science.

Of course, the mathematical equations that we use in statistics are exact. If you plug in the same numbers, you'll get the same results every time. 

But equations have to be used in the right circumstances, in order to yield useful results. If you use the wrong equation in the wrong circumstance, you might get results that look like they might be useful, but they are really not. Equations are dumb, in the sense that if you give them the wrong numbers, they'll still compute an answer for you. They don't know that you're using them wrong! 

The point here is just this. It takes good judgment to know when you should use one equation over another, and it takes understanding to know why.

In these notes, we will spend time understanding not only \emph{what} the equations do, but \emph{why}. We will focus on understanding how and why the equations work the way they do, so that we can understand what is really going on under the hood. This goes a long way towards developing good statistical judgment.


\end{document}
