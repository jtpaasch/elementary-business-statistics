\documentclass[../../../main.tex]{subfiles}
\begin{document}

%%%%%%%%%%%%%%%%%%%%%%%%%%%%%%%%%%%%%%%%%
%%%%%%%%%%%%%%%%%%%%%%%%%%%%%%%%%%%%%%%%%
%%%%%%%%%%%%%%%%%%%%%%%%%%%%%%%%%%%%%%%%%
\chapter{Introduction to inferential statistics}

At this point, we can begin our study of \vocab{inferential statistics}. Inferential statistics is the set of techniques that we use to collect information about a sample, and then use that information to make an \emph{estimate} about the larger population. 

We call it \emph{inferential} statistics because we use information we collect from samples to \emph{infer} something about the larger population.

With inferential statistics, we usually work with populations that are too big to measure directly. For example, we might be dealing with all people in the United States who are ages 30--40, or we might be dealing with all vehicles of a certain make and model. Whatever the population we are interested in, it is too big for us to measure it directly.

Often:

\begin{enumerate}

  \item We don't know the relevant parameters of the population. For example, suppose we want to know the average (mean) height of all high school students in the state of Delaware. It would be infeasible to track down every high school student in Delaware, measure their height, and compute the mean height of all those students. So, we simply don't (and probably never will) know the mean height of that population.
  
  \item We have to study samples, because we can't measure the entire population. Since we can't measure the height of every high school student in Delaware, we have to rely on samples. So, we are typically studying samples, in an attempt to estimate a parameter of the larger population.
  
\end{enumerate}

\end{document}
