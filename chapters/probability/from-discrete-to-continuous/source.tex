\documentclass[../../../main.tex]{subfiles}
\begin{document}

%%%%%%%%%%%%%%%%%%%%%%%%%%%%%%%%%%%%%%%%%
%%%%%%%%%%%%%%%%%%%%%%%%%%%%%%%%%%%%%%%%%
%%%%%%%%%%%%%%%%%%%%%%%%%%%%%%%%%%%%%%%%%
\chapter{From discrete to continuous}

Previously, we looked at discrete random variables, and discrete \PDFtext/s. Those were for discrete values. We have the same for continuous values. We have continuous random variables, and continuous \PDFtext/s. However, we have to do things a little differently to handle continuous values.


%%%%%%%%%%%%%%%%%%%%%%%%%%%%%%%%%%%%%%%%%
%%%%%%%%%%%%%%%%%%%%%%%%%%%%%%%%%%%%%%%%%
\section{Discrete vs continuous values}

Here are some properties of discrete values:

\begin{itemize}
  \item Discrete values don't have fractions or decimals between them.
  \item They are whole values. You can't get part of them.
\end{itemize}

Continuous values are not like this. 

\begin{itemize}
  \item Continuous values have parts.
  \item They can always be broken up into smaller fractions/decimals.
\end{itemize}


%%%%%%%%%%%%%%%%%%%%%%%%%%%%%%%%%%%%%%%%%
\subsection{Some examples}

Here are some examples of \vocab{discrete} values: 

\begin{itemize}
    \item Heads or tails (when you flip a coin). There's only two possible values here. You can't get  part of heads and part of tails. (Assume that if a coin manages to land on its edge, it is disqualified.)
    \item The numbers 1 through 6 (when you roll a six-sided die). There's only these six values. You can't get part of one and part of another. You can't roll 1.346, for example. (Assume that a die that doesn't land squarely on a side is disqualified.)
    \item The number of marbles in a bag. You can't get part of a marble. (Assume that the marbles are not shattered or broken into pieces. They are whole.)
    \item The number of points scored in a basketball game. You don't earn half points, or 0.264 of a point, in basketball. 
    \item The counting numbers (0, 1, 2, 3, and so on). 
\end{itemize}

Here are some examples of continuous values:

\begin{itemize}
    \item Mass. Objects needn't have a mass that falls exactly on whole numbers. They can have a mass that falls anywhere in between whole numbers. An object can have a mass of (say) 3.2 grams, or 3.735 grams, or 3.3462823734832 grams, and so on. 
    \item Height. Objects needn't have heights that fall exactly on whole numbers. A height can fall anywhere in between whole numbers. Something can have a height of 52.7 cm, or 52.35734 cm, or 52.00000000000234 cm, and so on.
    \item Time: Times needn't fall exactly on whole numbers either. For instance, if we measure how long something lasts, it needn't be exactly 3, or exactly 4 seconds. It could be anywhere in between: 3.2347834 seconds, or 3.99999934 seconds, or 3.0000234324 seconds, and so on.
\end{itemize}

Note that sometimes we want to treat continuous values as if they were discrete. For example, if we want to study how many miles people drive to work, we might want just a whole number (we want 20 miles, not 20.203463 miles). So, it depends on the experiment. It is always important to think carefully about the experiment itself, and make sure you understand it, so you know what sorts of values you are dealing with.


%%%%%%%%%%%%%%%%%%%%%%%%%%%%%%%%%%%%%%%%%
%%%%%%%%%%%%%%%%%%%%%%%%%%%%%%%%%%%%%%%%%
\section{From discrete to continuous}

As we saw, a discrete probability distribution function breaks up the probability into discrete chunks. Consider the following examples:

\begin{itemize}
  \item Flip a coin? There are exactly 2 outcomes. So we divide the total (which is 1, or 100\%) into two equal chunks, so that each of the outcomes gets exactly half. Hence, the probability of each outcome is .5.
  \item Roll a six-sided die? There are 6 outcomes, divided the total into six equal chunks, so that each outcome gets exactly one-sixth of the total. Hence, the probability of each outcome is 0.167.
\end{itemize}

Notice that the more outcomes you have, the lower the probability of getting any particular one. 

\begin{itemize}
  \item With a coin flip, you have a 50/50 chance, because there are only two outcomes. If you flip the coin a second time, the chances of flipping the same thing again isn't terribly bad.
  \item With a die roll, you have a one-in-six chance to get any particular outcome, because there are six outcomes. If you roll the die again, the changes of rolling the same thing are lower, but still not terrible. It is one out of 36. 
\end{itemize}

Now let's start thinking about cases where we have more and more outcomes, and see what happens. Imagine a 100-sided (fair) die.

\begin{itemize}
  \item There are 100 outcomes, which divide the total into 100 equal chunks, so that each outcome gets exactly one-hundredth.
  \item The probability of getting any one particular outcome is pretty small. And imagine rolling the die again. What are the chances that you would roll the same thing again? Now it's very low (one in ten thousand).
\end{itemize}

Imagine a one million sided (fair) die.

\begin{itemize}
  \item There are a million outcomes, which divide the total into one million equal chunks.
  \item Now the probability of getting any one particular outcome is really, really small.
  \item And imagine rolling the die a second time. The probability of rolling the same thing you rolled the first time is so low that it is practically impossible (the probability of rolling the same thing again would be one in a trillion).
\end{itemize}

We can really see now that the more outcomes that are possible, the lower the probability of getting any particular one of them is.

So, let's finally imagine that we have continuous values. 

\begin{itemize}
  \item Think about that. It's not just a million possible outcomes. There are infinitely many possible values. 
  \item Suppose we're just measuring objects that have a mass between 3 and 4 grams. There are an infinite number of masses between just 3.01 and 3.02 grams. There is 3.01000001 grams, and 3.01000002 grams, and 3.01002343243 grams, and so on.
\end{itemize}

What is the probability of getting any particular one of these values? 

\begin{itemize}
  \item Well, the probability is one over infinity. And really, we just treat this as zero. 
  \item This may seem unintuitive at first, but if you think about it, it makes a lot of sense. If there are an infinite number of possible outcomes, the chances of getting any particular one of them is practically non-existent (i.e., the probability of getting a particular one of them is basically zero).
  \item Think about the probability of an object being, say, exactly 3.015 grams. The chances of that are practically non-existent. Even if the object happened to be near 3.015 grams, it is very unlikely that it would be \emph{exactly} 3.015 grams. It might be somewhere nearer 3.014, or nearer 3.016, or nearer 3.01523463, or nearer any number of other masses that are extremely close to 3.015, but not exactly the same as 3.015. 
\end{itemize}

So with continuous values, we consider the probability of any particular value/outcome to always be zero. 




\end{document}
